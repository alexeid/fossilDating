% latex table generated in R 3.0.3 by xtable 1.7-4 package
% Sun Aug  9 15:25:34 2015
\begin{table}[ht]
\centering
\begin{tabular}{rrrrrrrr}
  \hline
 & post & bf & err & est & hpd\_lower & hpd\_upper & ess \\ 
  \hline
Anthropornis grandis & 0.92 & 83.0 & 5.1 & 38.2 & 32.3 & 45.3 & 288 \\ 
  Anthropornis nordenskjoeldi & 0.92 & 88.4 & 5.3 & 38.0 & 32.2 & 45.3 & 317 \\ 
  Archaeospheniscus lopdelli & 0.57 & 51.0 & 0.6 & 27.4 & 22.1 & 32.3 & 607 \\ 
  Archaeospheniscus lowei & 0.53 & 44.3 & 0.5 & 27.5 & 21.2 & 33.1 & 616 \\ 
  Burnside Palaeudyptes & 0.54 & 76.2 & 0.6 & 36.6 & 32.3 & 40.0 & 448 \\ 
  Delphinornis arctowskii & 0.37 & 13.0 & 5.3 & 42.8 & 32.8 & 53.0 & 142 \\ 
  Delphinornis gracilis & 0.18 & 4.7 & 7.7 & 45.2 & 35.5 & 53.6 & 401 \\ 
  Delphinornis larseni & 0.95 & 139.7 & 3.1 & 40.2 & 32.5 & 49.9 & 361 \\ 
  Delphinornis wimani & 0.83 & 37.9 & 6.2 & 37.1 & 25.0 & 45.8 & 719 \\ 
  Duntroonornis parvus & 0.78 & 63.4 & 0.3 & 26.1 & 18.8 & 33.4 & 273 \\ 
  Eretiscus tonnii & 0.56 & 39.2 & 1.9 & 16.6 & 10.9 & 21.7 & 873 \\ 
  Icadyptes salasi & 0.25 & 34.6 & 1.5 & 34.9 & 29.8 & 38.2 & 556 \\ 
  Inkayacu paracasensis & 0.34 & 53.5 & 0.3 & 36.2 & 31.1 & 39.4 & 857 \\ 
  Kairuku grebneffi & 0.68 & 84.5 & 0.8 & 28.8 & 24.6 & 32.7 & 950 \\ 
  Kairuku waitaki & 0.62 & 62.5 & 1.2 & 29.2 & 25.2 & 34.1 & 976 \\ 
  Madrynornis mirandus & 0.02 & 6.5 & 3.5 & 6.5 & 0.9 & 12.0 & 432 \\ 
  Marambiornis exilis & 0.75 & 67.0 & 1.3 & 38.8 & 32.0 & 45.3 & 447 \\ 
  Marplesornis novaezealandiae & 0.52 & 15.0 & 5.1 & 15.7 & 7.7 & 22.5 & 468 \\ 
  Mesetaornis polaris & 0.72 & 55.6 & 1.6 & 39.1 & 32.4 & 47.5 & 470 \\ 
  Pachydyptes ponderosus & 0.28 & 41.0 & 1.4 & 33.9 & 28.7 & 37.4 & 1478 \\ 
  Palaeeudyptes antarcticus & 0.18 & 7.8 & 4.3 & 36.6 & 30.2 & 41.6 & 400 \\ 
  Palaeeudyptes gunnari & 0.90 & 66.4 & 6.1 & 37.9 & 31.9 & 42.0 & 274 \\ 
  Palaeeudyptes klekowskii & 0.84 & 38.8 & 6.1 & 37.1 & 30.8 & 41.8 & 248 \\ 
  Palaeospheniscus bergi & 0.96 & 354.5 & 2.7 & 18.0 & 14.4 & 21.2 & 378 \\ 
  Palaeospheniscus biloculata & 0.78 & 111.2 & 0.8 & 17.7 & 13.5 & 21.4 & 421 \\ 
  Palaeospheniscus patagonicus & 0.87 & 205.1 & 0.6 & 17.9 & 14.6 & 20.9 & 306 \\ 
  Paraptenodytes antarcticus & 0.02 & 1.3 & 6.1 & 28.1 & 23.4 & 33.0 & 752 \\ 
  Perudyptes devriesi & 0.05 & 1.9 & 9.0 & 49.0 & 40.6 & 57.5 & 735 \\ 
  Platydyptes marplesi & 0.80 & 85.7 & 2.3 & 24.2 & 20.9 & 27.7 & 795 \\ 
  Platydyptes novaezealandiae & 0.58 & 72.6 & 0.1 & 24.4 & 20.5 & 28.8 & 527 \\ 
  Pygoscelis grandis & 0.77 & 85.1 & 1.3 & 4.3 & 0.3 & 8.0 & 697 \\ 
  Spheniscus megaramphus & 0.72 & 107.1 & 0.4 & 7.8 & 3.9 & 10.5 & 473 \\ 
  Spheniscus muizoni & 0.01 & 5.7 & 3.8 & 5.3 & 1.9 & 8.7 & 1466 \\ 
  Spheniscus urbinai & 0.60 & 60.4 & 1.5 & 9.2 & 5.3 & 11.7 & 276 \\ 
  Waimanu manneringi & 0.04 & 5.6 & 4.3 & 56.8 & 50.2 & 61.9 & 1010 \\ 
  Waimanu tuatahi & 0.43 & 25.7 & 2.2 & 60.5 & 53.7 & 65.6 & 918 \\ 
   \hline
\end{tabular}
\caption{Table of 36 fossil penguins. $post$ is the posterior probability that the phylogenetic age is within the paleaontological age range. Error is the difference in millions of years between the phylogenetic point estimate of the fossil's age and the mean of it's paleaontological age range. {\em Age} is the phylogenetic estimate of the age, along with the upper and lower of the corresponding 95\% HPD credible interval.} 
\label{fossilTable}
\end{table}
